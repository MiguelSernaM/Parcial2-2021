\documentclass{article}
\usepackage[utf8]{inputenc}
\usepackage[spanish]{babel}
\usepackage{listings}
\usepackage{graphicx}
\graphicspath{ {images/} }
\usepackage{cite}

\begin{document}

\begin{titlepage}
    \begin{center}
        \vspace*{1cm}
            
        \Huge
        \textbf{Analisis y diseño}
            
        \vspace{0.5cm}
        \LARGE
        Parcial 2
            
        \vspace{1.5cm}
        \textbf{Miguel Angel Serna Montoya}
            
        \vfill
            
        \vspace{0.8cm}
            
        \Large
        Departamento de Ingeniería Electrónica y Telecomunicaciones\\
        Universidad de Antioquia\\
        Medellín\\
        Septiembre de 2021
            
    \end{center}
\end{titlepage}

\tableofcontents

\section{Conociendo el problema} 
Más sin embargo en un principio el problema me resulto demasiado difícil de comprender y ni digamos lo difícil que va a ser programarlo. El día de hoy estuve buscando información acerca de como funcionaban las matrices leds RGB. 

\section{Análisis problema} \label{contenido}
1er día: Por el momento mis ideas algorítmicamente hablando son casi nulas por no decir obsoletas ya que no he podido dedicarle el merecido tiempo al planteamiento del problema. Por el momento me estoy encargando de entender bien los requisitos y restricciones con los que me veo comprometido.
\section{Consideraciones}
No puedo usar las librerías para el submuestreo o sobremestrueo. Todo tiene que ser de mi puño y letra.
\section{Diseño algoritmo}
\section{Esquema de tareas}

\end{document}